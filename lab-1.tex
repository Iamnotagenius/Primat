\documentclass{article}
\usepackage[a4paper,margin=5em]{geometry}
\usepackage{amsmath,amssymb}
\usepackage[utf8]{inputenc}
\usepackage[T2A]{fontenc}
\usepackage[english,russian]{babel}
\begin{document}

\begin{center}
    Национальный исследовательский университет ИТМО \\
    Факультет информационных технологий и программирования \\
    Прикладная математика и информатика
\end{center}
\vspace{20em}
\begin{center}
    {\Large Численное дифференцирование и интегрирование}
    \vspace{3pt}
    \hrule
    \vspace{3pt}
    Отчёт по лабораторной работе №1
\end{center}
\vspace{20em}
\begin{flushright}
\textbf{ Работу выполнили: } \\
Казаков Андрей \\
Обиджанов Алишер \\
Кузнецов Павел \\
\vspace{1em}
\textbf{ Преподаватель: } \\
Корсун М. М.
\end{flushright}
\vspace{12em}
\begin{center}
    Санкт-Петербург \\
    2021
\end{center}
\newpage

Выберем какую-нибудь функцию $f(x)$, определённую на отрезке $[a, b]$.
Отрезок можно разбить на $n = \frac{b-a}{h}$, и определить $x_i = a + hi$, $y_i = f(x_i)$.
Реализуем несколько операций над этой функцией.

\section{Реализация нахождения методов производной при фиксированном шаге}

Аналитически производная определяется следующим образом:
$$f'(x)=\lim_{h \to 0} \frac{f(x+h)-f(x)}{h}$$
В силу аппаратных ограничений мы не можем вычислить этот предел с математической точностью.
Что можно сделать, так это зафиксировать какое-то малое значение $h$.
Определим $h$ как \textit{шаг сетки}. Отсюда мы сразу же получаем первую формулу:
$$f'(x) \approx \frac{f(x+h)-f(h)}{h}$$
Данная формула называется \textit{правой разностной производной}. Примем во внимание тот факт,
что в общем случае, при приближении $h$ к нулю, $h$ необязательно быть положительным числом.
Таким образом мы получаем \textit{левую разностную производную}:
$$f'(x) \approx \frac{f(x)-f(x-h)}{h}$$
Для увеличения точности можно использовать формулу с тремя узлами:
$$f'(x) = \frac{f(x+h) - f(x-h)}{2h}$$
Значения производной в крайних точках определим следующим образом:
\begin{align}
    f'(a) &= \frac{-3f(a) + 4f(a + h) - f(a + 2h)}{2h} \\
    f'(b) &= \frac{3f(b - 2h) - 4f(b - h) + f(b)}{2h}
\end{align}

\end{document}
