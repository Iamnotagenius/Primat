\documentclass{article}
\usepackage[a4paper,margin=5em]{geometry}
\usepackage{amsmath,amssymb,tikz,pgfplots}
\usepackage[linkcolor=blue,colorlinks]{hyperref}
\usepackage[utf8]{inputenc}
\usepackage[T2A]{fontenc}
\usepackage[english,russian]{babel}
\pgfplotsset{
    compat = 1.18,
    every axis/.append style = {
        axis lines = middle,
        grid = both,
        minor tick num = 1]
    },
    every axis plot/.append style = {
        samples = 1000,
        smooth
    },
    minor grid style = {
        very thin
    }
}
\begin{document}

\begin{center}
    Национальный исследовательский университет ИТМО \\
    Факультет информационных технологий и программирования \\
    Прикладная математика и информатика
\end{center}
\vspace{20em}
\begin{center}
    {\Large Численное дифференцирование и интегрирование}
    \vspace{3pt}
    \hrule
    \vspace{3pt}
    Отчёт по лабораторной работе №1
\end{center}
\vspace{20em}
\begin{flushright}
\textbf{ Работу выполнили: } \\
Казаков Андрей \\
Обиджанов Алишер \\
Кузнецов Павел \\
\vspace{1em}
\textbf{ Преподаватель: } \\
Корсун М. М.
\end{flushright}
\vspace{12em}
\begin{center}
    Санкт-Петербург \\
    2021
\end{center}
\newpage

Выберем какую-нибудь функцию $f(x)$, определённую на отрезке $[a, b]$.
Отрезок можно разбить на $n = \frac{b-a}{h}$, и определить $x_i = a + hi$, $y_i = f(x_i)$.
Реализуем несколько операций над этой функцией.

\section{Реализация методов нахождения производной при фиксированном шаге}

Аналитически производная определяется следующим образом:
$$f'(x)=\lim_{h \to 0} \frac{f(x+h)-f(x)}{h}$$
В силу аппаратных ограничений мы не можем вычислить этот предел с математической точностью.
Что можно сделать, так это зафиксировать какое-то малое значение $h$.
Определим $h$ как \textit{шаг сетки}. Отсюда мы сразу же получаем первую формулу:
$$f'(x) \approx \frac{f(x+h)-f(h)}{h}$$
Данная формула называется \textit{правой разностной производной}. Примем во внимание тот факт,
что в общем случае, при приближении $h$ к нулю, $h$ необязательно быть положительным числом.
Таким образом мы получаем \textit{левую разностную производную}:
$$f'(x) \approx \frac{f(x)-f(x-h)}{h}$$
Для увеличения точности можно использовать формулу с тремя узлами:
$$f'(x) = \frac{f(x+h) - f(x-h)}{2h}$$
Значения производной в крайних точках определим следующим образом:
\begin{align}
    f'(a) &= \frac{-3f(a) + 4f(a + h) - f(a + 2h)}{2h} \\
    f'(b) &= \frac{3f(b - 2h) - 4f(b - h) + f(b)}{2h}
\end{align}

\def\Fi{x^2}
\def\Fii{e^x}

Выберем две функции: $f_1(x) = \Fi, f_2(x) = \Fii$. Будем находить производные
на отрезке $[0, 1]$ с шагом $0.1$. И вычислять для каждого значения среднеквадратичное отклонение $\sigma$ от аналитически высчитаного.

\begin{figure}[htpb]
    \begin{center}
        \begin{tikzpicture}
            \begin{axis}[restrict x to domain = 0:1,
                         legend entries = {$\Fi$, $\Fii$},
                         legend pos = south east,
                         yminorgrids = false,
                         ymajorgrids = false,
                         no marks,
                         grid style = { dashed }]
                \addplot {x^2};
                \addplot {exp(x)};
            \end{axis}
        \end{tikzpicture}
    \end{center}
    \caption{Графики выбранных функций}%
\end{figure}

\begin{table}[htp]
    \centering
    \caption{Вычисленные значения производных функции $\Fi$}
    \label{tab:Fi}
    \begin{tabular}{r|c|c|c|c|c}
        $x$ & Лев. разн. & Прав. разн. & Три узла & Аналитически & $\sigma$ \\
        \hline
        0.0 & -0.1 & 0.1 & 0.0 & 0.0 & 0.08165 \\
        0.1 & 0.1 & 0.3 & 0.2 & 0.2 & 0.08165 \\
        0.2 & 0.3 & 0.5 & 0.4 & 0.4 & 0.08165 \\
        0.3 & 0.5 & 0.7 & 0.6 & 0.6 & 0.08165 \\
        0.4 & 0.7 & 0.9 & 0.8 & 0.8 & 0.08165 \\
        0.5 & 0.9 & 1.1 & 1.0 & 1.0 & 0.08165 \\
        0.6 & 1.1 & 1.3 & 1.2 & 1.2 & 0.08165 \\
        0.7 & 1.3 & 1.5 & 1.4 & 1.4 & 0.08165 \\
        0.8 & 1.5 & 1.7 & 1.6 & 1.6 & 0.08165 \\
        0.9 & 1.7 & 1.9 & 1.8 & 1.8 & 0.08165 \\
        1.0 & 1.9 & 2.1 & 2.0 & 2.0 & 0.08165 \\
        \hline
    \end{tabular}
\end{table}

\begin{table}[htp]
    \centering
    \caption{Вычисленные значения производных функции $\Fii$}
    \label{tab:Fii}
    \begin{tabular}{r|c|c|c|c|c}
        $x$ & Лев. разн. & Прав. разн. & Три узла & Аналитически & $\sigma$ \\
        \hline
        0.0 & 0.95163 & 1.05171 & 0.99640 & 1.00000 & 0.04093 \\
        0.1 & 1.05171 & 1.16232 & 1.10701 & 1.10517 & 0.04519 \\
        0.2 & 1.16232 & 1.28456 & 1.22344 & 1.22140 & 0.04995 \\
        0.3 & 1.28456 & 1.41966 & 1.35211 & 1.34986 & 0.05520 \\
        0.4 & 1.41966 & 1.56897 & 1.49431 & 1.49182 & 0.06100 \\
        0.5 & 1.56897 & 1.73398 & 1.65147 & 1.64872 & 0.06742 \\
        0.6 & 1.73398 & 1.91634 & 1.82516 & 1.82212 & 0.07451 \\
        0.7 & 1.91634 & 2.11788 & 2.01711 & 2.01375 & 0.08235 \\
        0.8 & 2.11788 & 2.34062 & 2.22925 & 2.22554 & 0.09101 \\
        0.9 & 2.34062 & 2.58679 & 2.46370 & 2.45960 & 0.10058 \\
        1.0 & 2.58679 & 2.85884 & 2.70987 & 2.71828 & 0.11123 \\
        \hline
    \end{tabular}
\end{table}

\begin{figure}[htpb]
    \begin{center}
        \begin{tikzpicture}
            \begin{axis}[xlabel = $n$,
                ylabel = $\sigma$,
                legend entries={$\sigma_{\Fi}$, $\sigma_{\Fii}$}]
                \addplot table {
                        10 0.08164965809277268
                        20 0.040824829046386284
                        40 0.020412414523193083
                        80 0.01020620726159653
                        160 0.005103103630798458
                    };
                \addplot table {
                        10 0.07085295201908055
                        20 0.03523355290230071
                        40 0.017574973571553184
                        80 0.008777788902433966
                        160 0.0043865604063861915
                    };
            \end{axis}
        \end{tikzpicture}
    \end{center}
    \caption{График зависимости среднего квадратичного от количества шагов}%
    \label{fig:dev}
\end{figure}

В таблицах \ref{tab:Fi} и \ref{tab:Fii} левые разностные производные меньше,
чем аналитически посчитанные, в то же время правые разностные производные - больше.
Метод трёх узлов точнее вычисляет производную засчёт того, что мы рассматриваем
значения производных в двух направлениях окрестности точки.

Как можно заметить на графике \ref{fig:dev}, зависимость $\sigma$ от $n$ обратная.
Учитывая, что $n = \frac{b-a}{h}$, следует, что $\sigma$ линейно зависит от $h$.

\section{Реализация методов численного интегрирования}

Для нахождения приближенного значения определенного интеграла могут ис- пользоваться так называемые квадратурные формулы
$$I=\int_a^b f(x)dx\backsimeq \sum_0^n A_i f(\bar x_i),$$
где $\bar x_i-$ некоторые точки из отрезка $[a,b]$.
\vspace{5mm}

Введём также сетку узлов на отрезке таким же образом,
тогда интеграл $I$ разобьётся в сумму элементарных интегралов:
$$I=\sum_1^nI_i,$$
где каждый $I_i$ вычисляется на отрезке $[x_{i-1},x_i]$. Геометрически это будет означать, что вся криволинейная трапеция разбивается на $n$ элементарных криволинейных трапеций. Методы численного интегрирования отличаются способом вычисления площадей этих элементарных криволинейных трапеций.

1. Формула прямоугольников. Площадь каждой элементарной криволинейной трапеции можно приближать площадью прямоугольников. Причем в зависимости от той точки, которая определяет высоту прямоугольника можно получить либо метод левых прямоугольников:
$$I-i \backsimeq hf_{i-1}$$
либо правых прямоугольников:
$$I-i \backsimeq hf_{i}$$
либо средних прямоугольнков:
$$I-i \backsimeq hf_{i-1/2}$$
2. Формула трапеций. Используя оба конца отрезка элементарной криволинейной трапеции, можно приближать ее площадь как площадь трапеции
$$I_i \backsimeq \frac{h}{2}(f_{i-1}+f_i)$$
3. Формула Симпсона. Также криволинейную трапецию можно приближать параболой, которая проходит
соответственно через точки $x_{i-1}$, $x_{i-1/2}$ и $x_i$. Таким образом
$$I_i = \frac{h}{6}(f_{i-1} + 4f_{i-1/2}+f_i)$$

Выберем две функции: $f_1(x) = \Fi, f_2(x) = \Fii$. Будем находить интегралы определенные
на отрезке $[0, 1]$ с шагом $0.1$ и вычислять для каждого найденного значения среднеквадратичное отклонение $\sigma$ от аналитически высчитаного.

\begin{figure}[htpb]
    \begin{center}
        \begin{tikzpicture}
            \begin{axis}
            [restrict x to domain = 0:1,
                         legend entries = {$\Fi$, $\Fii$},
                         legend pos = south east,
                         yminorgrids = false,
                         ymajorgrids = false,
                         no marks,
                         grid style = { dashed }]
                \addplot {x^2};
                \addplot {exp(x)};
            \end{axis}
        \end{tikzpicture}
    \end{center}
    \caption{Графики выбранных функций}%
\end{figure}

\begin{table}[htp]
    \centering
    \caption{Вычисленные разными методами значения интегралов функций}
    \label{tab:ints}
    \begin{tabular}{1000, r|c|c|c|c|c|c}
        $f(x)$ & Лев. пр. & Прав. пр. & Сред. пр. & Трап. & Симп. & Аналит.\\
        \hline
        $\Fi$ & 0.28500 & 0.38500 & 0.33250 & 0.33500 & 0.35000 & 0.33333 \\
        $\Fii$ & 1.63380 & 1.80563 & 1.71757 & 1.71971 & 1.74692 & 1.71828 \\
        \hline
    \end{tabular}
\end{table}

\begin{table}[htp]
    \centering
    \caption{Вычисленные интегралы и среднекв. отклонение при разном шаге. для $\Fi$}
    \label{tab:ints1}
    \begin{tabular}{1200, r|r|r|r|r|r|r}
        $n$ & Лев. пр. & Прав. пр. & Сред. пр. & Трап. & Симп. & $\sigma$.\\
        \hline
        10 & 0.28500 & 0.38500 & 0.33250 & 0.33500 & 0.35000 & 0.03252 \\
        20 & 0.30875 & 0.35875 & 0.33313 & 0.33375 & 0.34167 & 0.01625 \\
        40 & 0.32094 & 0.34594 & 0.33328 & 0.33344 & 0.33750 & 0.00812 \\
        80 & 0.32711 & 0.33961 & 0.33332 & 0.33336 & 0.33542 & 0.00406 \\
        160 & 0.33021 & 0.33646 & 0.33333 & 0.33334 & 0.33437 & 0.00203 \\
        \hline
    \end{tabular}
\end{table}

\begin{table}[htp]
    \centering
    \caption{Вычисленные интегралы и среднекв. отклонение при разном шаге. для $\Fii$}
    \label{tab:ints2}
    \begin{tabular}{1200, r|r|r|r|r|r|r}
        $n$ & Лев. пр. & Прав. пр. & Сред. пр. & Трап. & Симп. & $\sigma$.\\
        \hline
        10 & 1.63380 & 1.80563 & 1.71757 & 1.71971 & 1.74692 & 0.05584 \\
        20 & 1.67568 & 1.76160 & 1.71810 & 1.71864 & 1.73260 & 0.02791 \\
        40 & 1.69689 & 1.73985 & 1.71824 & 1.71837 & 1.72544 & 0.01396 \\
        80 & 1.70756 & 1.72904 & 1.71827 & 1.71830 & 1.72186 & 0.00698 \\
        160 & 1.71292 & 1.72366 & 1.71828 & 1.71829 & 1.72007 & 0.00349 \\
        \hline
    \end{tabular}
\end{table}

\begin{figure}[htpb]
    \begin{center}
        \begin{tikzpicture}
            \begin{axis}[xlabel = $n$,
                ylabel = $\sigma$,
                legend entries={$\sigma_{\Fi}$, $\sigma_{\Fii}$}]
                \addplot table {
                        10 0.03252
                        20 0.01625
                        40 0.00812
                        80 0.00406
                        160 0.00203
                    };
                \addplot table {
                        10 0.05584
                        20 0.02791
                        40 0.01396
                        80 0.00698
                        160 0.00349
                    };
            \end{axis}
        \end{tikzpicture}
    \end{center}
    \caption{График зависимости среднего квадратичного от количества шагов}%
    \label{fig:dev}
\end{figure}

\newpage

По таблице \ref{tab:ints} видно, что значения интегралом посчитанные методом левых прямоугольников меньше аналитических значений, в то время как значения посчитанные методом правых прям-уг. наоборот больше, что в целом выходит из формул подсчёта приближенных значений. Наиболее близким по значениям оказался метод сред. прямоугольников, оказываясь как-бы чем-то средним между указанными раннее двумя методами.
Формула трапеций при данных экспериментальных данных оказалась довольно близка к значениям интегралов высчитанных методом сред. прямоугольников, в то время как значения полученные при помощи формулы Симпсона оказались больше аналитически посчитанных значений.

Метод трёх узлов точнее вычисляет производную засчёт того, что мы рассматриваем
значения производных в двух направлениях окрестности точки.

Как можно заметить на графике \ref{fig:dev}, зависимость $\sigma$ от $n$ обратная.
Учитывая, что $n = \frac{b-a}{h}$, следует, что $\sigma$ линейно зависит от $h$.

\end{document}
