\documentclass{article}
\usepackage[a4paper,margin=1.3cm]{geometry}
\usepackage{amsmath,amssymb,setspace,longtable,multirow,tikz,pgfplots,array,xcolor}
\usepackage{minted}
\usepackage[utf8]{inputenc}
\usepackage[T2A]{fontenc}
\usepackage[english,russian]{babel}
\usepackage{graphicx}
\usepackage{listings}
\setlength{\parindent}{2em}
\newcommand{\row}{\therow\addtocounter{row}{1}}
\newcounter{row}
\setcounter{row}{1}
\pgfplotsset{compat=1.16}
\newcolumntype{L}{>{\centering\arraybackslash}m{3cm}}
\definecolor{bg}{rgb}{0.96,0.96,0.93}
\setminted{frame=lines,framesep=1em}
\begin{document}
\begin{center}
    Национальный исследовательский университет ИТМО\\
    Факультет информационных технологий и программирования\\
    Прикладная математика
\end{center}
\vspace{20em}
\begin{center}
    {\Large Теория игр}
    \vspace{3pt}
    \hrule
    \vspace{3pt}
    Отчет по лабораторной работе №2
\end{center}
\vspace{20em}
\begin{flushright}
    \textbf{ Работу выполнили: } \\
    Обиджанов Алишер\\
    Казаков Андрей\\
    Кузнецов Павел\\
    \vspace{1em}
    \textbf{ Преподаватель: } \\
    Свинцов М.В.
\end{flushright}
\vspace{12em}
\begin{center}
    Санкт-Петербург \\
    2023
\end{center}
\newpage

\section{Постановка задачи}

\begin{enumerate}
    \item Реализуйте возможность ввода данных из файла в формате JSON, который
          содержит матрицу игры.
    \item Упростите платежную матрицу путем анализа доминирующих стратегий.
    \item Если это возможно найдите решение игры в чистых стратегиях. Определите
          оптимальные стратегии и соответствующую цену игры.
    \item Если решение в чистых стратегиях найти невозможно, примените симплекс-
          метод для поиска седловой точки в смешанных стратегиях. Определите сме-
          шанные стратегии и соответствующую цену игры
\end{enumerate}

\section{Реализация}

\subsection{Реализуйте возможность ввода данных из файла в формате JSON, который содержит матрицу игры.}

Пример json строки:
\begin{listing}[H]
    \begin{minted}{json}
    {"matrix":[[10,4,11,7],[7,6,8,20],[6,2,1,11]]}
    \end{minted}
\end{listing}

Парсинг строки:

\begin{listing}[H]
    \begin{minted}{python}
        def parse_problem(json_str):
            try:
                data = json.loads(json_str)

                if "matrix" in data:
                    matrix = data["matrix"]

                    if isinstance(matrix, list) and all(isinstance(line, list) for line in matrix):
                        return np.array(matrix)
                    else:
                        raise ValueError("Неверный формат матрицы в JSON.")
                else:
                    raise ValueError("Отсутствует поле 'matrix' в JSON.")

            except json.JSONDecodeError as e:
                raise ValueError(f"Ошибка декодирования JSON: {e}")
        \end{minted}
\end{listing}

\subsection{Упростите платежную матрицу путем анализа доминирующих стратегий}

Сперва разделим каждый элемент на максимальный общий делитель и избавимся от отрицательных значений

\begin{listing}[H]
    \begin{minted}{python}
       def simplefy_matrix(matrix):
            if matrix.dtype != np.float64:
                matrix //= np.gcd.reduce(matrix.flatten())
            
            matrix -= matrix.min() if matrix.min() < 0 else 0

            return matrix
    \end{minted}
\end{listing}

Далее будем сокращать строки и столбцы путем анализа доминирующих стратегий до тех пор пока это возможно

\begin{listing}[H]
    \begin{minted}{python}
    def reduce_matrix(matrix):
    abbreviated = True
    deleteLines = []
    deleteColumn = []

    while abbreviated:
        ilines = list(set(range(matrix.shape[0]))-set(deleteLines))
        icolumns = list(set(range(matrix.shape[1]))-set(deleteColumn))

        abbreviated = False
        for iline1 in ilines:
            for iline2 in ilines:
                if iline1 != iline2 and np.all(matrix[iline1,icolumns] > matrix[iline2,icolumns]):
                    deleteLines.append(iline2)
                    # print("delete line:",iline2)
                    abbreviated = True
                    break
        

        for icolumn1 in icolumns:
            for icolumn2 in icolumns:
                if icolumn1 != icolumn2 and np.all(matrix[ilines,icolumn1] > matrix[ilines,icolumn2]):
                    deleteColumn.append(icolumn1)
                    # print("delete column:",icolumn1)
                    abbreviated = True
                    break

    matrix = np.delete(matrix, deleteColumn, axis=1)
    matrix = np.delete(matrix, deleteLines, axis=0)
    return matrix, sorted(deleteLines), sorted(deleteColumn)

    \end{minted}
\end{listing}

\subsection{Если это возможно найдите решение игры в чистых стратегиях. Определите оптимальные стратегии и соответствующую цену игры.}

В чистых стратегиях можно решить задачу только в том случае, если у матрицы есть седловая точка. Ее можно найти путем вычисления минимаксных значений. Так же её можно найти, если мы сократили нашу изначальную матрицу до размера 1:1. Использую второй способ.

\begin{listing}[H]
    \begin{minted}{python}
    def strategies(r_matrix, deleteLines, deleteColumn):
    Pa = []
    Qb = []

    if r_matrix.shape[0] == 1 and r_matrix.shape[1] == 1:
        for i in range(len(deleteLines) + 1):
            Qb.append(0) if i in deleteLines else Qb.append(1)
        for i in range(len(deleteColumn) + 1):
            Pa.append(0) if i in deleteColumn else Pa.append(1)
    else:
        f = np.ones(r_matrix.shape[1])
        b = np.ones(r_matrix.shape[0])
        result = linprog(c=f, A_eq=r_matrix, b_eq=b)
        if result.success:
            strategy_2 = result.x
            strategy_1 = r_matrix.dot(strategy_2)
            game_value = 1 / result.fun
            strategy_2 = [x / sum(strategy_2) for x in strategy_2]
            strategy_1 = [x / sum(strategy_1) for x in strategy_1]
            k=0
            for i in range(len(deleteLines) + len(strategy_1)):
                if i in deleteLines: Pa.append(0)
                else:
                    Pa.append(strategy_1[k])
                    k+=1
            k=0
            for i in range(len(deleteColumn) + len(strategy_2)):
                if i in deleteColumn: Qb.append(0)
                else:
                    Qb.append(strategy_2[k])
                    k+=1

    return np.array(Pa), np.array(Qb)
    \end{minted}
\end{listing}

Вычислить цену игры можно по формуле $Q_b^T * M * P_a$

\begin{listing}[H]
    \begin{minted}{python}
        def game_price(matrix, Pa, Qb):
        return Qb.T @ matrix @ Pa
    \end{minted}
\end{listing}

\subsection{Запуск кода}
\begin{listing}[H]
    \begin{minted}{python}
    json_string = '{"matrix":[[-2,1],[2,-1]]}'
    try:
        matrix = parse_problem(json_string)
        matrix = simplefy_matrix(matrix)
        r_matrix, deleteLines, deleteColumn = reduce_matrix(matrix)
        print(f"reduced matrix={r_matrix}, delete lines:{deleteLines}, delete column:{deleteColumn}")

        Pa, Qb = strategies(r_matrix, deleteLines, deleteColumn)
        y = game_price(matrix, Pa, Qb)

        print(f"Pa={Pa}, Qb={Qb}, y={y}")
        
    except ValueError as e:
        print(f"Ошибка: {e}")
    \end{minted}
\end{listing}

\section{Вывод}

В ходе данной лабораторной работы был успешно реализован алгоритм для решения матричной игры, предназначенной для двух игроков. Проведен анализ сценариев как в чистых стратегиях, так и в смешанных. Для эффективного решения матрицы в смешанных стратегиях применен симплекс-метод. Полученные стратегии обеспечивают оптимальные решения, а использование симплекс-метода добавляет методу дополнительную гибкость и применимость к различным сценариям.

\end{document}