\documentclass{article}
\usepackage[a4paper,margin=1.3cm]{geometry}
\usepackage{amsmath,amssymb,setspace,longtable,multirow,tikz,pgfplots,array,xcolor}
\usepackage{minted}
\usepackage[utf8]{inputenc}
\usepackage[T2A]{fontenc}
\usepackage[english,russian]{babel}
\usepackage{graphicx}
\usepackage{listings}
\setlength{\parindent}{2em}
\newcommand{\row}{\therow\addtocounter{row}{1}}
\newcounter{row}
\setcounter{row}{1}
\pgfplotsset{compat=1.16}
\newcolumntype{L}{>{\centering\arraybackslash}m{3cm}}
\definecolor{bg}{rgb}{0.96,0.96,0.93}
\setminted{frame=lines,framesep=1em}
\begin{document}
\begin{center}
    Национальный исследовательский университет ИТМО\\
    Факультет информационных технологий и программирования\\
    Прикладная математика
\end{center}
\vspace{20em}
\begin{center}
    {\Large Теория игр}
    \vspace{3pt}
    \hrule
    \vspace{3pt}
    Отчет по лабораторной работе №2
\end{center}
\vspace{20em}
\begin{flushright}
    \textbf{ Работу выполнили: } \\
    Обиджанов Алишер\\
    Казаков Андрей\\
    Кузнецов Павел
    \vspace{1em}
    \textbf{ Преподаватель: } \\
    Свинцов М.В.
\end{flushright}
\vspace{12em}
\begin{center}
    Санкт-Петербург \\
    2023
\end{center}
\newpage

\section{Постановка задачи}

\begin{enumerate}
    \item Реализуйте возможность ввода данных из файла в формате JSON, который
          содержит матрицу игры.
    \item Упростите платежную матрицу путем анализа доминирующих стратегий.
    \item Если это возможно найдите решение игры в чистых стратегиях. Определите
          оптимальные стратегии и соответствующую цену игры.
    \item Если решение в чистых стратегиях найти невозможно, примените симплекс-
          метод для поиска седловой точки в смешанных стратегиях. Определите сме-
          шанные стратегии и соответствующую цену игры
\end{enumerate}

\section{Реализация}

\subsection{Реализуйте возможность ввода данных из файла в формате JSON, который содержит матрицу игры.}

Пример json строки:
\begin{listing}[H]
    \begin{minted}{json}
    {"matrix":[[10,4,11,7],[7,6,8,20],[6,2,1,11]]}
    \end{minted}
\end{listing}

Парсинг строки:

\begin{listing}[H]
    \begin{minted}{python}
        def parse_problem(json_str):
            try:
                data = json.loads(json_str)

                if "matrix" in data:
                    matrix = data["matrix"]

                    if isinstance(matrix, list) and all(isinstance(line, list) for line in matrix):
                        return np.array(matrix)
                    else:
                        raise ValueError("Неверный формат матрицы в JSON.")
                else:
                    raise ValueError("Отсутствует поле 'matrix' в JSON.")

            except json.JSONDecodeError as e:
                raise ValueError(f"Ошибка декодирования JSON: {e}")
        \end{minted}
\end{listing}

\subsection{Упрастите платежную матрицу путем анализа доминирующих стратегий}

Сперва разделим каждый элемент на максимальный общий делитель и избавимся от отрицательных значений

\begin{listing}[H]
    \begin{minted}{python}
        def simplefy_matrix(matrix):
        min_modul_of_number = 1e9
        min_number = 1e9

        for line in matrix:
            for num in line:
                if abs(num) < min_modul_of_number and num != 0:
                    min_modul_of_number = abs(num)
                if num < min_number:
                    min_number = num
        
        findDel = True

        for line in matrix:
            if not findDel:
                break
            for num in line:
                if num % min_modul_of_number != 0:
                    findDel = False
                    break
        
        if findDel:
            min_number /= min_modul_of_number
            for i in range(matrix.shape[0]):
                for j in range(matrix.shape[1]):
                    matrix[i][j] = matrix[i][j] / min_modul_of_number
        
        if min_number < 0:
            for i in range(matrix.shape[0]):
                for j in range(matrix.shape[1]):
                    matrix[i][j] += -min_number
        return matrix
    \end{minted}
\end{listing}

Далее будем сокращать строки и столбцы путем анализа доминирующих стратегий до тех пор пока это возможно

\begin{listing}[H]
    \begin{minted}{python}
    def reduce_matrix(matrix):
    abbreviated = True
    deleteLines = []
    deleteColumn = []

    while abbreviated:
        ilines = list(set(range(matrix.shape[0]))-set(deleteLines))
        icolumns = list(set(range(matrix.shape[1]))-set(deleteColumn))

        abbreviated = False
        for iline1 in ilines:
            for iline2 in ilines:
                is_reduce = True
                if iline1 == iline2:
                    continue
                for column in icolumns:
                    if not matrix[iline1][column] >= matrix[iline2][column]:
                        is_reduce = False
                        break
                if not is_reduce:
                    continue
                else:
                    deleteLines.append(iline2)
                    abbreviated = True
        
        for icolumn1 in icolumns:
            for icolumn2 in icolumns:
                is_reduce = True
                if icolumn1 == icolumn2:
                    continue
                for iline in ilines:
                    if not matrix[iline][icolumn1] >= matrix[iline][icolumn2]:
                        is_reduce = False
                        break
                if not is_reduce:
                    continue
                else:
                    deleteColumn.append(icolumn1)
                    abbreviated = True

    matrix = np.delete(matrix, deleteColumn, axis=1)
    matrix = np.delete(matrix, deleteLines, axis=0)
    return matrix, sorted(deleteLines), sorted(deleteColumn)
    \end{minted}
\end{listing}

\subsection{Если это возможно найдите решение игры в чистых стратегиях. Определите оптимальные стратегии и соответствующую цену игры.}

В чистых стратегиях можно решить задачу только в том случае, если у матрицы есть седловая точка. Ее можно найти путем вычисления минимаксных значений. Так же её можно найти, если мы сократили нашу изначальную матрицу до размера 1:1. Использую второй способ.

\begin{listing}[H]
    \begin{minted}{python}
    def strategies(matrix, deleteLines, deleteColumn):
    Pa = []
    Qb = []

    if matrix.shape[0] == 1 and matrix.shape[1] == 1:
        for i in range(len(deleteLines) + 1):
            Pa.append(0) if i in deleteLines else Pa.append(1)
        for i in range(len(deleteColumn) + 1):
            Qb.append(0) if i in deleteColumn else Qb.append(1)
    return Pa, Qb
    \end{minted}
\end{listing}
\end{document}