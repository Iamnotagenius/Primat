\documentclass[fleqn]{article}
\usepackage[a4paper,margin=5em]{geometry}
\usepackage{amsmath,amssymb,tikz,pgfplots}
\usepackage[linkcolor=blue,colorlinks]{hyperref}
\usepackage[utf8]{inputenc}
\usepackage[T2A]{fontenc}
\usepackage[english,russian]{babel}
\pgfplotsset{
    compat = 1.18,
    every axis/.append style = {
        axis lines = middle,
        grid = both,
        minor tick num = 1]
    },
    every axis plot/.append style = {
        samples = 1000,
        smooth
    },
    minor grid style = {
        very thin
    }
}

\begin{document}
    \begin{center}
    Национальный исследовательский университет ИТМО \\
    Факультет информационных технологий и программирования \\
    Прикладная математика и информатика
\end{center}
\vspace{20em}
\begin{center}
    {\Large Методы оптимизации}
    \vspace{3pt}
    \hrule
    \vspace{3pt}
    Отчёт по лабораторной работе №2
\end{center}
\vspace{20em}
\begin{flushright}
\textbf{ Работу выполнили: } \\
Казаков Андрей \\
Обиджанов Алишер \\
Кузнецов Павел \\
\vspace{1em}
\textbf{ Преподаватель: } \\
Корсун М. М.
\end{flushright}
\vspace{12em}
\begin{center}
    Санкт-Петербург \\
    2021
\end{center}

\newpage

\section*{Одномерная оптимизация:}
\subsection*{Постановка задачи:}
\begin{align*}
    &f(x) \rightarrow \underset{x}{min}, x \in R\\
    &f: R \rightarrow R\\
    &\underset{2[a,b]}{min}(z=f(x))
\end{align*}

\begin{itemize}
    \item [$f(x):$]
    \begin{itemize}
        \item Неприрывана
        \item Унимодальна
    \end{itemize}
\end{itemize}

Определение: Унимод на $[a, b]$  $ \exists x_* \in [a, b]$:  $f(x) [a,x_*)$ убывает, $f(x) (x_*, b]$  возрастает\\

Точность: $\varepsilon$

\subsection*{Вариант задачи:}
Горный хребет: $\sin{x}*x^3$

\subsection*{Решение:}
\subsubsection{Метод дихотомии}

\begin{itemize}
    \item[1)] Находим середину промежутка графика:\\
    $$\frac{a+b}{2}$$\\
    
    \item[2)] Величина отступа:\\
    $\varepsilon$ - точность\\
    $\delta$ - отступ\\
    $$\delta < \frac{\varepsilon}{2} $$\\
    
    \item[3)] Находим значения $f(x_1)$ и $f(x_2)$:\\
    $$x_1 = \frac{a_i+b_i}{2} - \delta$$\\
    $$x_2 = \frac{a_i+b_i}{2}+\delta$$\\
    
    \item[4)] Выбераем промежуток для следующей итерации\\
    Если $x_2>x_1$ : $[a, x_2],\space b_1=x_2$\\
    Если $x_1>x_2$ : $[x_1, b],\space a_1=x_1$\\
\end{itemize}

Интервал неопределённости на первой итерации:\\
$$a_2b_1 \cong a_0b_0/2$$\\
$$n:\space a_nb_n \approx \frac{b_0-a_0}{2^n}$$\\

Для того что бы достичь точность $\varepsilon$ нам потребуется столько итераций:
$$\ln{((b_0-a_0)/\varepsilon)/\ln{2}}$$

\end{document}