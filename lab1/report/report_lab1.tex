\documentclass{article}
\usepackage[a4paper,margin=1.3cm]{geometry}
\usepackage{amsmath,amssymb,setspace,longtable,multirow,tikz,pgfplots,array,xcolor}
\usepackage{minted}
\usepackage[utf8]{inputenc}
\usepackage[T2A]{fontenc}
\usepackage[english,russian]{babel}
\usepackage{graphicx}
\setlength{\parindent}{2em}
\newcommand{\row}{\therow\addtocounter{row}{1}}
\newcounter{row}
\setcounter{row}{1}
\pgfplotsset{compat=1.16}
\newcolumntype{L}{>{\centering\arraybackslash}m{3cm}}
\definecolor{bg}{rgb}{0.96,0.96,0.93}
\setminted{frame=lines,framesep=1em}

\begin{document}
\begin{center}
    Национальный исследовательский университет ИТМО\\
    Факультет информационных технологий и программирования\\
    Прикладная математика
\end{center}
\vspace{20em}
\begin{center}
    {\Large — Линейное программирования}
    \vspace{3pt}
    \hrule
    \vspace{3pt}
    Отчет по лабораторной работе №1
\end{center}
\vspace{20em}
\begin{flushright}
    \textbf{ Работу выполнили: } \\
    Обиджанов Алишер\\
    Какзаков Андрей\\
    Кузнецов Павел\\
    \vspace{1em}
    \textbf{ Преподаватель: } \\
    Свинцов М.В.
\end{flushright}
\vspace{12em}
\begin{center}
    Санкт-Петербург \\
    2023
\end{center}
\newpage

\section{Постановка задачи}
\begin{enumerate}
    \item Реализуйте возможность ввода данных из файла в формате JSON. Рекоменду-
          емая структура JSON указана ниже.
    \item При необходимости добавьте балансирующие переменные для перехода от об-
          щей постановки к канонической форме задачи линейного программирования.
    \item Реализуйте симплекс-метод для решения задачи.
    \item Предусмотрите, что задача как может не иметь решений вообще, так и иметь
          бесконечное количество решений
\end{enumerate}

\section{Реализация}

\subsection{Подключаем библиотеки}

\begin{listing}[H]
    \begin{minted}{python}
    import numpy as np
    from json import loads
    \end{minted}
\end{listing}

\subsection{Подготавливаем пример в формате json}
\begin{listing}[H]
    \begin{minted}{python}
    example = """{"f": [1, 2, 3],
              "goal": "max",
              "constraints": [{"coefs": [1, 0, 0],
                                "type": "eq",
                                "b": 1},
                               {"coefs": [1, 1, 0],
                                "type": "gte",
                                "b": 2},
                               {"coefs": [1, 1, 1],
                                "type": "lte",
                                "b": 3}]}"""
    \end{minted}
\end{listing}

\subsection{Парсим json файл}

Приобразует json в матрицы f, A и b, которые затем возвращает. Если задача имеет цель "min", то целевая функция домножается на -1.

\begin{listing}[H]
    \begin{minted}{python}
    def parse_problem(json: str):
    parsed = loads(json)
    f = np.array(parsed['f'])
    if parsed['goal'] == "min":
        f *= -1
    A = []
    b = []
    for constraint in parsed['constraints']:
        A.append(np.array(constraint['coefs']))
        b.append(constraint['b'])
        if constraint['type'] == "gte":
            A[-1] *= -1
            b[-1] *= -1
        if constraint['type'] == "eq":
            A.append(np.array(constraint['coefs']) * -1)
            b.append(-constraint['b'])
    A = np.array(A)
    A = np.hstack((A, np.eye(A.shape[0])))
    b = np.array(b)
    b = b.reshape((b.shape[0], 1))
    return f, A, b
    \end{minted}
\end{listing}

\subsection{Симплекс-метод}

Сначала создается таблица tableau, которая объединяет матрицы A и b. Если в таблице есть отрицательные элементы в столбце b, то выбирается строка i с минимальным значением b и столбец l с минимальным значением в этой строке. Затем находятся отношения между элементами строки i и столбцом l и выбирается строка r с наименьшим отношением. Далее выполняется шаг симплекс-метода для выбранных строк и столбцов.

\subsubsection{Шаг симплекс-метода}
\begin{listing}[H]
    \begin{minted}{python}
    def simplex_step(tableau, r, l):
    for i in range(tableau.shape[0]):
        if i == r:
            tableau[i] /= tableau[i, l]
            continue
        tableau[i] -= tableau[r] * tableau[i, l] / tableau[r, l]
    \end{minted}
\end{listing}

\subsubsection{Симплекс-метод}
\begin{listing}[H]
    \begin{minted}{python}
    def simplex(f, A, b):
    tableau = np.hstack((b, A))
    tableau = np.vstack((tableau, np.hstack((np.zeros((1,)), f, np.zeros((A.shape[0]))))))
    print(tableau)
    if b[b < 0].any():
        i = tableau[:-1, 0].argmin()
        l = tableau[i, 1:].argmin()
        if tableau[i, l] >= 0:
            raise Exception("No solution.")
        ratios = tableau[:-1, 0] / tableau[:-1, l]
        r = ratios.argmin()
        simplex_step(tableau, r, l)
        print(tableau)
    s = tableau[-1].argmax()

    last_max = -1
    while s > 0:
        print(tableau[:-1, s])
        temp = tableau[:-1, s]
        temp[temp == 0] = -1
        ratios = tableau[:-1, 0] / temp
        j = ratios.argmin()
        simplex_step(tableau, j, s)
        print(tableau)
        s = tableau[-1].argmax()
        if tableau[-1, s] == last_max:
            break
        last_max = tableau[-1, s]
    return tableau[-1, 0]
    \end{minted}
\end{listing}

\section{Результат}
\begin{listing}[H]
    \begin{minted}{python}
    print(simplex(*parse_problem(example)))
    \end{minted}
\end{listing}

\textbf{Output:}

$$\begin{pmatrix}
1& 1& 0& 0& 1& 0& 0& 0\\
0& 0& 0& 0& 1& 1& 0& 0\\
0& 1&-1& 0& 2& 0& 1& 0\\
0&-2& 1& 1&-3& 0& 0& 1\\
0& 1& 2& 3& 0& 0& 0& 0
\end{pmatrix}$$

$$\begin{pmatrix}
0&0&0&1
\end{pmatrix}$$

$$\begin{pmatrix}
-1&-1&-0& 1&-1&-0&-0&-0\\
-1&-1& 0& 0& 0& 1& 0& 0\\
-1& 0&-1& 0& 1& 0& 1& 0\\
1&-1& 1& 0&-2& 0& 0& 1\\
3& 4& 2& 0& 3& 0& 0& 0
\end{pmatrix}$$

$$\begin{pmatrix}
-1&-1& 0&-1
\end{pmatrix}
$$

$$\begin{pmatrix}
-2& 0&-1& 1& 1&-0&-0&-1\\
-2& 0&-1& 0& 2& 1& 0&-1\\
-2& 0&-2& 0& 3& 0& 1&-1\\
-1& 1&-1&-0& 2&-0&-0&-1\\
7& 0& 6& 0&-5& 0& 0& 4
\end{pmatrix}$$

$$7.0$$
\end{document}