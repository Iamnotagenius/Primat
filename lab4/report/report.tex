\documentclass{article}
\usepackage[margin=5em]{geometry}
\usepackage{amsmath,amssymb,graphicx,svg}
\usepackage[utf8]{inputenc}
\usepackage[T2A]{fontenc}
\usepackage[english,russian]{babel}
\begin{document}
\hrule
\begin{center}
    \huge
    Лабораторная работа №4 \\
    \vspace{5pt}
    \normalsize
    \textit{Регрессия временных рядов}
\end{center}
\hrule
\begin{flushright}
    \textbf{Выполнил студент группы №M33071} \\
    Казаков Андрей Павлович
\end{flushright}
\section{Построение авторегрессионной модели}
\textbf{Авторегрессионная (AR- модель)} - модель временного ряда, в которой значения ряда зависят от предыдущих его значений. 
В нашем случае мы рассматриваем AR-3 модель такого вида:
$$ x_t = a_0 + \sum_{i=1}^{3} a_i x_{t - i} + \epsilon_t $$
Где $a_i$ - коэффициенты (параметры) модели, $\epsilon_t$ - белый шум.

Нам необходимо рассмотреть стационарную модель, условие стационарности - все корни характеристического уравнения лежат вне единичной окружности.
Характеристическое уравнение имеет вид:
$$ 1 - \sum_{i=1}^3 a_i z^i = 0 $$

\section{Демонстрация регрессии ряда}
Воспользовавшись генератором авторегрессионных моделей, мы получили пример следующего ряда:
$$x_t = 0.326 + 0.938x_{t-1} - 0.505x_{t-2} + 0.521x_{t-3} + \epsilon_t$$
График первых 1003 значений ряда с первыми тремя значениями $t_0 = 4.736, t_1 = 6.022, t_2 = -0.006$:

\begin{figure}[h]
    \centering
    \includesvg[width=38em]{../process.svg}
\end{figure}

Как можно наблюдать из графика, ряд быстро сходится к некоторому "равновесию" вокруг значения 0, то есть ряд обладает свойством стационарности.
Колебания зависят сугубо от заданной дисперсии.
\end{document}
